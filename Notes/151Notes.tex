%%% Template originaly created by Karol Kozioł (mail@karol-koziol.net) and modified for ShareLaTeX use

\documentclass[a4paper,11pt]{article}

\usepackage[T1]{fontenc}
\usepackage[utf8]{inputenc}
\usepackage{graphicx}
\usepackage{xcolor}

\usepackage{tgtermes}

\usepackage{listings}
\usepackage{color}

\definecolor{dkgreen}{rgb}{0,0.6,0}
\definecolor{gray}{rgb}{0.5,0.5,0.5}
\definecolor{mauve}{rgb}{0.58,0,0.82}

\lstset{frame=tb,
  language=Java,
  aboveskip=3mm,
  belowskip=3mm,
  showstringspaces=false,
  columns=flexible,
  basicstyle={\small\ttfamily},
  numbers=none,
  numberstyle=\tiny\color{gray},
  keywordstyle=\color{blue},
  commentstyle=\color{dkgreen},
  stringstyle=\color{mauve},
  breaklines=true,
  breakatwhitespace=true,
  tabsize=3
}

\usepackage[
pdftitle={Problem Set No.9}, 
pdfauthor={James Capuder, Oberlin College},
colorlinks=true,linkcolor=blue,urlcolor=blue,citecolor=blue,bookmarks=true,
bookmarksopenlevel=2]{hyperref}
\usepackage{amsmath,amssymb,amsthm,textcomp}
\usepackage{enumerate}
\usepackage{multicol}
\usepackage{tikz}

\usepackage{geometry}
\geometry{total={210mm,297mm},
left=25mm,right=25mm,%
bindingoffset=0mm, top=20mm,bottom=20mm}


\linespread{1.3}

\newcommand{\linia}{\rule{\linewidth}{0.5pt}}

% custom theorems if needed
\newtheoremstyle{mytheor}
    {1ex}{1ex}{\normalfont}{0pt}{\scshape}{.}{1ex}
    {{\thmname{#1 }}{\thmnumber{#2}}{\thmnote{ (#3)}}}

\theoremstyle{mytheor}
\newtheorem{defi}{Definition}

% my own titles
\makeatletter
\renewcommand{\maketitle}{
\begin{center}
\vspace{2ex}
{\huge \textsc{\@title}}
\vspace{1ex}
\\
\linia\\
\@author \hfill \@date\
\vspace{4ex}
\end{center}
}
\makeatother
%%%

% custom footers and headers
\usepackage{fancyhdr,lastpage}
\pagestyle{fancy}
\lhead{}
\chead{}
\rhead{}
\lfoot{Notes}
\cfoot{}
\rfoot{Page \thepage\ /\ \pageref*{LastPage}}
\renewcommand{\headrulewidth}{0pt}
\renewcommand{\footrulewidth}{0pt}
%

%%%----------%%%----------%%%----------%%%----------%%%

\begin{document}

\title{CSCI 151 Data Structures - Notes}

\author{James Capuder, Oberlin College}

\date{\today}

\maketitle

\section*{2/4/15 - Intro to Java}
\subsection*{Why Java?}
\begin{itemize}
\item Speed: Compiler vs interpreter 
\item Platform independence: Java runs on virtual machine, which is the same on every platform (Windows, OSX, Android)
\item Can send program without source code
\item Memory Management (Garbage Collection)
\item Pure object-oriented language
\item Strongly Typed Language
\end{itemize}
\subsection*{Hello World Notes}
\begin{itemize}
\item File name has to be identical to main class
\item Octothorp to comment in python
\item double forward slash to line comment in Java
\item Block comment, everything between forwardslash star to star forwardslash, even works across multiple lines
\item Finally, variation of the block comment, the Javadoc comment, goes from forwardslash doublestar to star forwardslash.
\end{itemize}
\subsection*{Whitespace}
\begin{itemize}
\item lines end in ';'
\item BLOCKs are surrounded by \{\}
\item strips away all whitespace
\end{itemize}
\subsection*{Visibility}
\begin{itemize}
\item Public, anyone can access
\item Private, only I can access the variable, along with other variables in the same class
\item protected
\end{itemize}
\subsection*{Variables}
\begin{itemize}
\item In Java, variables must be declared with a type and visibility
\end{itemize}

\subsubsection*{Primitive Types}
\begin{tabular}{l|cc}
  Types& Description & \\
  \hline
  Byte & 8 bit integer & \\
  Short& 16 bit integer & \\
  Int& 32 bit integer & \\
  Long & 64 bit integer & \\
  Float& 32 bit floating point number & \\
  Double& 64 bit floating point number & \\
  Char& 16 bit unicode character & \\
  Boolean& true/false &
\end{tabular}

\subsubsection*{Non-primative types/objects}
\begin{tabular}{l|cc}
  Objects& Description & \\
  \hline
  String & Character String, double quotes & \\
  Arrays& TBC &
\end{tabular}

\subsubsection*{Operators}
\begin{tabular}{l|cc}
  Operator& Function & \\
  \hline
  = & assignment & \\
  + & addition &\\
  - & subtraction &\\
  * & multiplication &\\
  / & division &\\
  \% & modulus division &\\
  int y = a/b & integer division &\\
  f=(double) a/b & cast a into a double &
\end{tabular}

\subsection*{Arrays}
\begin{itemize}
\item int a[]; - declares a variable a to be an array of integers
\item a = newint[4]; - creates an array with 4 slots
\item int a[] = \{ 2,4,8,16 \}; - does both of the above steps and assigns values to the indices
\end{itemize}

\subsection*{Logical Operators}
\begin{tabular}{l|cc}
  Operator& Function & \\
  \hline
  == & equality & \\
  != & not equal &\\
  dobule ampersand & and &\\
  || & or &\\
   ! & not &\\
   wedge symbol (shift 6) & exclusive or &
\end{tabular}

\subsection*{Loops and decision points}
\subsubsection*{if statements}
\begin{lstlisting}
if (<test>) {
    statement;
    otherstatement;
} else {
    this;
    that;
}
something else;
\end{lstlisting}
\subsubsection*{Elif statements}
\begin{lstlisting}
if (<test>) {
    statement;
} else if (othertest) {
    otherstatement;
} else {
    lastone;
}
\end{lstlisting}

\subsubsection*{for loops}
\begin{lstlisting}
for ( <init> ; <test> ; <incr> ) {
    command;
    cmd2;
}
for (int i=0 ; i<10 ; i+=1 ) {
    command;
    cmd2;
}
public static void main(String[] args) {
    for (string s: args) {
        System.out.println(s);
    }
}
\end{lstlisting}
\subsubsection*{pre and post increment operators}
\begin{lstlisting}
int a=3;
a++; // a is now 4 (post increment)
++a; // a is now 5 (pre-increment)
int b, c;
a = 5;
b = a++; // a is now 6, b is now 5
a = 5;
b = ++a // a is now 6, b is now 6

\end{lstlisting}






\clearpage



\end{document}
